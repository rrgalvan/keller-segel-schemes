\documentclass[a4wide]{article}
\usepackage{amsmath}
\usepackage[margin={3cm,2cm}]{geometry}
\usepackage{xcolor}

% Some definitons, in order to make more easy PDE expressions
\newcommand{\um}{u^m}
\newcommand{\umm}{u^{m+1}}
\newcommand{\vm}{v^m}
\newcommand{\vmm}{v^{m+1}}
\newcommand{\grad}{\nabla}
\renewcommand{\div}{\nabla\cdot}
\newcommand{\nn}{\mathbf{n}}

% Highlight Math
\newcommand{\hm}[1]{\colorbox{yellow}{\ensuremath{#1}}}

\title{Descrizione del primo schema de Keller-Segel}

\begin{document}
\maketitle

\section{Descrizione}

\begin{itemize}
\item Schema contenuto nell'articolo di Giuseppe.
\item Time
  semi-discretization: we fix time steps $t_m=k\cdot m$
  ($m=0,1,2,...$) and given inital data $u^0,v^0 \in
  V=H^1(\Omega)$.
  For each find $m\ge 0$, we search $\umm$ and $\vmm$ such that
  \begin{align}
    \label{eq:ks.1}
    (1/k) \umm - \Delta \umm \hm{+ k_1 \div(\umm \grad \vm)}&= (1/k) \um ,
    \\
    \label{eq:ks.2}
    (1/k) \vmm - \Delta \vmm + k_2 \vmm - k_3 \umm &= (1/k) \vm .
  \end{align}
  As in other schema, we assume Neumann boundary conditions for each unknown:
  $\grad \umm \cdot \nn=0$, $\grad \vmm \cdot \nn=0$, $m\ge 0$.
  Color \hm{\qquad\rule{0pt}{1ex}} highlights differences with previous Schema.
\end{itemize}

\section{Validation}
Here we propose some ideas to validate the software
which we are developing (using \texttt{FreeFEM++}).

\subsection*{Test 1. Reproduction of results contained in Giuseppe's paper}

Idea: if we use the data contained in this paper (inital data $u^0$
and $v^0$, time step, $k_i$ parameters...), we must obtain the same
results. For instance, blow-up in the time step which is reflected in
graphics contained in that paper (\textit{write here more details!}).

\subsection*{Test 2. Comparison with exact solution}

Idea:
\begin{itemize}
\item To compute a exact solution, $(u,v)$, to (a \textit{modified version} of)
  Keller-Segel equations
\item Use (a \textit{modified version} of)
  scheme~\eqref{eq:ks.1}--\eqref{eq:ks.2} and finite elements to
  aproximate the solution, $(\um_h, \vm_h)$.
\item Compute errors $\|u-\um_h\|_{L^2(\Omega)}$,
  $\|v-\vm_h\|_{L^2(\Omega)}$. When $k\to 0$ and $h\to 0$, errors must
  vanish.
\end{itemize}

\subsection*{Test 3. Discrete energy law}

Idea: analyze the discrete energy law of
scheme~\eqref{eq:ks.1}--\eqref{eq:ks.2}. Use \textit{FreeFEM++} to
compute this energy law. Plot the results and test if they are
agree with the previous analysis.

\end{document}
